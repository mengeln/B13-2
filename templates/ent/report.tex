\documentclass{article}\usepackage[]{graphicx}\usepackage[]{color}
%% maxwidth is the original width if it is less than linewidth
%% otherwise use linewidth (to make sure the graphics do not exceed the margin)
\makeatletter
\def\maxwidth{ %
  \ifdim\Gin@nat@width>\linewidth
    \linewidth
  \else
    \Gin@nat@width
  \fi
}
\makeatother

\definecolor{fgcolor}{rgb}{0.345, 0.345, 0.345}
\newcommand{\hlnum}[1]{\textcolor[rgb]{0.686,0.059,0.569}{#1}}%
\newcommand{\hlstr}[1]{\textcolor[rgb]{0.192,0.494,0.8}{#1}}%
\newcommand{\hlcom}[1]{\textcolor[rgb]{0.678,0.584,0.686}{\textit{#1}}}%
\newcommand{\hlopt}[1]{\textcolor[rgb]{0,0,0}{#1}}%
\newcommand{\hlstd}[1]{\textcolor[rgb]{0.345,0.345,0.345}{#1}}%
\newcommand{\hlkwa}[1]{\textcolor[rgb]{0.161,0.373,0.58}{\textbf{#1}}}%
\newcommand{\hlkwb}[1]{\textcolor[rgb]{0.69,0.353,0.396}{#1}}%
\newcommand{\hlkwc}[1]{\textcolor[rgb]{0.333,0.667,0.333}{#1}}%
\newcommand{\hlkwd}[1]{\textcolor[rgb]{0.737,0.353,0.396}{\textbf{#1}}}%

\usepackage{framed}
\makeatletter
\newenvironment{kframe}{%
 \def\at@end@of@kframe{}%
 \ifinner\ifhmode%
  \def\at@end@of@kframe{\end{minipage}}%
  \begin{minipage}{\columnwidth}%
 \fi\fi%
 \def\FrameCommand##1{\hskip\@totalleftmargin \hskip-\fboxsep
 \colorbox{shadecolor}{##1}\hskip-\fboxsep
     % There is no \\@totalrightmargin, so:
     \hskip-\linewidth \hskip-\@totalleftmargin \hskip\columnwidth}%
 \MakeFramed {\advance\hsize-\width
   \@totalleftmargin\z@ \linewidth\hsize
   \@setminipage}}%
 {\par\unskip\endMakeFramed%
 \at@end@of@kframe}
\makeatother

\definecolor{shadecolor}{rgb}{.97, .97, .97}
\definecolor{messagecolor}{rgb}{0, 0, 0}
\definecolor{warningcolor}{rgb}{1, 0, 1}
\definecolor{errorcolor}{rgb}{1, 0, 0}
\newenvironment{knitrout}{}{} % an empty environment to be redefined in TeX

\usepackage{alltt}

\usepackage{siunitx} % Provides the \SI{}{} command for typesetting SI units
\usepackage{graphicx} % Required for the inclusion of images
\usepackage[margin=1.0in]{geometry}
\setlength\parindent{0pt} % Removes all indentation from paragraphs
\renewcommand{\labelenumi}{\alph{enumi}.}


\title{qPCR Report \\ Enterococcus TaqEnviron Assay \\ 20111110 plate 13 env mm MR-ls.pcrd} % Title
\date{\today} % Date for the report
\author{}
\IfFileExists{upquote.sty}{\usepackage{upquote}}{}
\begin{document}

\maketitle

\begin{center}
\begin{tabular}{l r}
Organization: SCCWRP \\
Date Performed: 11/10/2011 12:34:47 PM UTC -08:00 \\ % Date the experiment was performed
Protocol: Sipp Entero1A.prcl \\
Sample Volume: 25 \si{\micro\litre}
\end{tabular}
\end{center}

%----------------------------------------------------------------------------

\section{Standard Curve QC Results}

Both enterococcus and sketa standard curves must have an $r^2$ that is greater than 0.98,
and an efficiency that is between 1.87 and 2.1. \\

% latex table generated in R 3.1.0 by xtable 1.7-3 package
% Tue Jun 10 15:12:43 2014
\begin{table}[ht]
\centering
\begin{tabular}{llrl}
  \hline
Target & Parameter & Value & QC \\ 
  \hline
Entero1A & r-squared & 1.00 & PASS \\ 
  Entero1A & Amplification Factor & 1.98 & PASS \\ 
  Sketa & r-squared & 1.00 & PASS \\ 
  Sketa & Amplification Factor & 1.98 & PASS \\ 
   \hline
\end{tabular}
\end{table}


%----------------------------------------------------------------------------

\section{NTC and NEC QC Results}

Both the NTCs (qPCR blanks) and NECs (extraction blanks) must be non-detects. Detectable signals
in any replicates will cause these tests to fail. \\

% latex table generated in R 3.1.0 by xtable 1.7-3 package
% Tue Jun 10 15:12:43 2014
\begin{table}[ht]
\centering
\begin{tabular}{lllll}
  \hline
Target & Sample & Ct$_{Rep1}$ & Ct$_{Rep2}$ & QC \\ 
  \hline
Entero1A & NEC & N/A & N/A & PASS \\ 
  Entero1A & NTC & N/A & N/A & PASS \\ 
  Sketa & NTC & N/A & N/A & PASS \\ 
   \hline
\end{tabular}
\end{table}


%----------------------------------------------------------------------------
\pagebreak
\section{Sample Processing and Inhibition Control QC Results}

The sketa calibrator internal control Ct on this plate was 21.82, with a standard deviation of NA. In order to pass,
the difference between the mean sample sketa Ct ($\text{sketaCt}_{mean}$) and the calibrator Ct (i.e., $\Delta \text{Ct}_{mean}$)
must be less than 1.7. Note that the threshold level and pass/fail designations assume that the sample has not
been diluted. \\

Additionally, the mean sketa Ct in the NECs was 22.08. A large difference between this value and the calibrator sketa would indicate some sort of problem.

% latex table generated in R 3.1.0 by xtable 1.7-3 package
% Tue Jun 10 15:12:43 2014
\begin{table}[ht]
\centering
\begin{tabular}{rrl}
  \hline
Calibrator Ct & $\Delta$ Ct & QC \\ 
  \hline
21.94 & -0.14 & PASS \\ 
  21.70 & -0.38 & PASS \\ 
   \hline
\end{tabular}
\end{table}


% latex table generated in R 3.1.0 by xtable 1.7-3 package
% Tue Jun 10 15:12:43 2014
\begin{table}[ht]
\centering
\begin{tabular}{lrrl}
  \hline
Sample & sketaCt$_{mean}$ & $\Delta$Ct$_{mean}$ & QC \\ 
  \hline
1 & 22.17 & 0.09 & PASS \\ 
  1 1:5 & 24.42 & 2.34 & FAIL \\ 
  2 & 21.98 & -0.10 & PASS \\ 
  2 1:5 & 24.44 & 2.36 & FAIL \\ 
  3 & 22.07 & -0.01 & PASS \\ 
  3 1:5 & 23.91 & 1.83 & FAIL \\ 
  4 & 22.13 & 0.05 & PASS \\ 
  4 1:5 & 24.09 & 2.02 & FAIL \\ 
  5 & 22.12 & 0.04 & PASS \\ 
  5 1:5 & 24.26 & 2.18 & FAIL \\ 
  6 & 35.88 & 13.80 & FAIL \\ 
  6 1:5 & 38.55 & 16.48 & FAIL \\ 
  7 & 22.25 & 0.17 & PASS \\ 
  7 1:5 & 24.16 & 2.08 & FAIL \\ 
  8 & 22.23 & 0.16 & PASS \\ 
  8 1:5 & 24.23 & 2.15 & FAIL \\ 
  9 & 21.98 & -0.10 & PASS \\ 
  9 1:5 & 24.19 & 2.11 & FAIL \\ 
   \hline
\end{tabular}
\end{table}


%----------------------------------------------------------------------------
\pagebreak
\section{Enterococcus Cell Equivalence Estimation}

Cell equivalents (CE) per reaction is calculated using the $\Delta$Ct method, in which samples are compared to the calibrator
in the following way:

$$ \log_{10}{CE} = \frac{\Delta\text{Ct}_{samp, cal}}{slope} + \log_{10}{cal} $$

where $slope$ is the slope of the enterococcus standard curve (\ensuremath{-3.365} for this plate),
$cal$ is the enterococcus calibrator expected CE, and
$\Delta\text{Ct}_{samp, cal}$ is the difference in Ct between the mean of the calibrators and
the sample. These values are then transformed to CE per filter (assumed to be 100 ml), and are reported below.
Samples indicated to be inhibited by sketa controls are labeled as such. Uninhibited CE estimates that are below the detection
limit (set to a Ct of 45) are denoted by ``ND''. Any detected inhibition among sample replicates causes the mean
to be labeled as ``inhibited''. Remember that these CE estimates (both replicate and mean) are reported on a logarithmic scale.

% latex table generated in R 3.1.0 by xtable 1.7-3 package
% Tue Jun 10 15:12:43 2014
\begin{table}[ht]
\centering
\scalebox{0.93}{
\begin{tabular}{lllllll}
  \hline
Sample & Target & Ct$_{Rep 1}$ & Ct$_{Rep 2}$ & $\log_{10}$ cells/100 \si{\milli\litre}$_{Rep1}$ & $\log_{10}$ cells/100 \si{\milli\litre}$_{Rep2}$ & Mean $\log_{10}$ cells/100 \si{\milli\litre} \\ 
  \hline
1 & Entero1A & 39.05 & 37.08 & 0.275 & 0.86 & 0.659 \\ 
  1 1:5 & Entero1A & N/A & 39.16 & N/A & 0.242 & Inhibited \\ 
  2 & Entero1A & 36.66 & 36.56 & 0.985 & 1.015 & 1 \\ 
  2 1:5 & Entero1A & N/A & N/A & N/A & N/A & Inhibited \\ 
  3 & Entero1A & N/A & 38.22 & N/A & 0.522 & N/A \\ 
  3 1:5 & Entero1A & N/A & 39.92 & N/A & 0.016 & Inhibited \\ 
  4 & Entero1A & 39.16 & 37.89 & 0.242 & 0.62 & 0.471 \\ 
  4 1:5 & Entero1A & N/A & N/A & N/A & N/A & Inhibited \\ 
  5 & Entero1A & 36.35 & 36.23 & 1.077 & 1.113 & 1.095 \\ 
  5 1:5 & Entero1A & N/A & N/A & N/A & N/A & Inhibited \\ 
  6 & Entero1A & N/A & N/A & N/A & N/A & Inhibited \\ 
  6 1:5 & Entero1A & N/A & N/A & N/A & N/A & Inhibited \\ 
  7 & Entero1A & N/A & 36.29 & N/A & 1.095 & N/A \\ 
  7 1:5 & Entero1A & 39.81 & 37.52 & 0.049 & 0.73 & Inhibited \\ 
  8 & Entero1A & N/A & 38.43 & N/A & 0.459 & N/A \\ 
  8 1:5 & Entero1A & N/A & 39.9 & N/A & 0.022 & Inhibited \\ 
  9 & Entero1A & 38.17 & 39.15 & 0.536 & 0.245 & 0.414 \\ 
  9 1:5 & Entero1A & N/A & N/A & N/A & N/A & Inhibited \\ 
   \hline
\end{tabular}
}
\end{table}


\end{document}
